\documentclass[unicode,bachelor]{scutthesis}

\usepackage[unicode=false,bookmarks=true,bookmarksnumbered=true,bookmarksopen=false,
 breaklinks=false,pdfborder={0 0 1},backref=false,colorlinks=true]
 {hyperref}
 
\usepackage{multirow}
 
\hypersetup{pdftitle={华南理工大学学位论文模板},
 pdfauthor={ShusenLiu},
 pdfsubject={华南理工大学学位论文模板},
 pdfkeywords={Online Advertising, Sponsed Search, Click Prediction, Consumer Desire},
 linkcolor=black, anchorcolor=black, citecolor=black, filecolor=black, menucolor=black, urlcolor=black, 
pdfstartview=FitH}

\begin{document}
\includepdf[pages=1,scale=1.0]{thesis_cover_scut.pdf}%包含pdf头文件

\frontmatter1
\begin{abstractCn}
中文摘要
\end{abstractCn}
\keywordsCn{中文关键词}

\begin{abstractEn}
Abstract in English
\end{abstractEn}
\keywordsEn{Keywords in English}


\tableofcontents{}
 
\mainmatter

\chapter{绪论}

\chapter{结论}

\bibliographystyle{scutthesis}
\bibliography{thesis}

\backmatter
\pagestyle{appendix_style}

\appendix{攻读本科学位期间取得的研究成果}

已发表(包括已接受待发表)的论文,以及已投稿、或已成文打算投稿、或拟成文投稿的论文情况(只填写与学位论文内容相关的部分):

\begin{table}
\begin{longtable}{|>{\centering}m{0.5cm}|>{\centering}m{2.3cm}|>{\centering}m{3.5cm}|>{\centering}m{2.6cm}|>{\centering}m{2cm}|>{\centering}m{1.3cm}|>{\centering}m{0.9cm}|}
\hline 
序号 & 作者(全体作者,按顺序排列) & 题 目 & 发表或投稿刊物名称、级别 & 发表的卷期、年月、页码 & 相当于学位论文的哪一部分(章、节) & 被索引收录情况\tabularnewline
\hline 
 &  &  &  &  &  & \tabularnewline
\hline 
 &  &  &  &  &  & \tabularnewline
\hline 
\end{longtable}
\end{table}
%\clearpage

\appendix{致谢}

%\begin{minipage}[t]{0.8\columnwidth}%
\begin{flushright}
ShusenLiu
%\par
\end{flushright}
\vspace{-30pt}
\begin{flushright}
2013 年 5 月 15 日
%\par
\end{flushright}%
%\end{minipage}
%\clearpage
\end{document}
